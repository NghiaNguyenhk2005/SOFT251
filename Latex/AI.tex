\section{Khai báo sử dụng AI trong dự án}

Trong quá trình thực hiện đồ án, nhóm có sử dụng công cụ Trí tuệ Nhân tạo (AI) 
như một nguồn hỗ trợ. Việc sử dụng AI được khai báo minh bạch và cụ thể như sau:

\begin{itemize}
    \item \textbf{Mục đích sử dụng:} AI được dùng để gợi ý ý tưởng, tham khảo tài liệu, 
    hỗ trợ viết nội dung báo cáo, và kiểm tra ngữ pháp/ngôn ngữ. 
    \item \textbf{Phạm vi đóng góp:} 
    \begin{enumerate}
        \item Gợi ý cấu trúc báo cáo, bố cục các phần mục và cách trình bày.
        \item Hỗ trợ tìm kiếm tài liệu tham khảo liên quan đến chủ đề dự án.
        \item Hỗ trợ kiểm tra ngữ pháp, chính tả và cải thiện câu văn.
        \item Đề xuất các công nghệ, framework phù hợp cho phát triển ứng dụng.
        \item Hỗ trợ xây dựng định hướng thiết kế hệ thống.
        \item Hỗ trợ phân tích yêu cầu chức năng và phi chức năng bằng cách đưa ra ví dụ tham khảo.
        \item Hỗ trợ viết nháp cho một số đoạn mô tả kỹ thuật (ReactJS, ExpressJS, Git/GitHub).
        \item Gợi ý ý tưởng xây dựng cho các biểu đồ UML cơ bản và mockup dựa trên mô tả của nhóm.
        \item Xây dựng một số đoạn mã mẫu minh họa ý tưởng (không phải mã hoàn chỉnh).
    \end{enumerate}
    \item \textbf{Giới hạn:} AI không được sử dụng để tạo ra toàn bộ nội dung báo cáo, 
    không thay thế cho phân tích và thiết kế của nhóm. Tất cả các phần quan trọng 
    (use-case, lược đồ UML, thiết kế lớp, mockup UI) đều do nhóm 
    tự xây dựng và chỉnh sửa.
    \item \textbf{Trách nhiệm:} Nhóm chịu trách nhiệm hoàn toàn về nội dung đã nộp, 
    đảm bảo hiểu rõ và làm chủ các phần có sự hỗ trợ từ AI. 
\end{itemize}

\noindent Việc sử dụng AI chỉ mang tính chất tham khảo và hỗ trợ, không thay thế cho 
năng lực phân tích, sáng tạo và đóng góp cá nhân của các thành viên trong nhóm.